\clearpage
\makeatletter
\efloat@restorefloats
\makeatother


\begin{appendix}
\renewcommand{\appendixname}{Supplementary Material}
\renewcommand{\thefigure}{SI\arabic{figure}} \setcounter{figure}{0}
\renewcommand{\thetable}{SI\arabic{table}} \setcounter{table}{0}
\renewcommand{\theequation}{SI\arabic{table}} \setcounter{equation}{0}

\hypertarget{forecasting}{%
\section{Forecasting}\label{forecasting}}

\hypertarget{forecasting-models}{%
\subsubsection{Forecasting models}\label{forecasting-models}}

\begin{itemize}
\tightlist
\item
  We did not assume that the existing processes would be the same for
  all drugs or all GP practices, therefore, we investigated the fit of
  several different time series models to the pre-COVID data.
\item
  We used the \texttt{fable} (O'Hara-Wild et al., 2021),
  \texttt{fasster} (O'Hara-Wild \& Hyndman, 2021), and the
  \texttt{fable.prophet} (O'Hara-Wild, 2020) packages to conduct the
  time series analyses.

  \begin{itemize}
  \tightlist
  \item
    Simple methods

    \begin{itemize}
    \tightlist
    \item
      Naïve (with and without drift)

      \begin{itemize}
      \tightlist
      \item
        \(\hat{y}_{T+h|T} = y_T\)
      \end{itemize}
    \item
      Seasonal naïve (with and without drift)

      \begin{itemize}
      \tightlist
      \item
        \(\hat{y}_{T+h|T} = y_{T-m(k+1)}\) where m = seasonal period and
        k is the complete years in \(T+h\)
      \end{itemize}
    \end{itemize}
  \item
    STL Decomposition model (Cleveland et al., 1990)

    \begin{itemize}
    \tightlist
    \item
      Fixed and variable seasonality
    \end{itemize}
  \item
    Time series linear model

    \begin{itemize}
    \tightlist
    \item
      \(y_t = \beta_0 + \beta_1x_t + \varepsilon_t\)
    \end{itemize}
  \item
    Autoregressive integrated moving average (ARIMA; Box et al., 2015)

    \begin{itemize}
    \tightlist
    \item
      Using a version of the Hyndman-Khandakar algorithm (Hyndman \&
      Khandakar, 2008)
    \end{itemize}
  \item
    Exponential smoothing (Holt, 2004; Winters, 1960)

    \begin{itemize}
    \tightlist
    \item
      Simple exponential smoothing
    \item
      Holt-Winters Additive Model (Chatfield, 1978)
    \end{itemize}
  \item
    Forecasting with Additive Switching of Seasonality, Trends and
    Exogenous Regressors (fasster; O'Hara-Wild \& Hyndman, 2021)

    \begin{itemize}
    \tightlist
    \item
      State space model
    \end{itemize}
  \item
    Prophet (Taylor \& Letham, 2018)
  \item
    Combination models (cf.~{\textbf{???}}; Clemen, 1989; Thomson et
    al., 2019)
  \end{itemize}
\item
  Prescribing quantities log transformed and forecasts use median values
  to reduce bias that back transformation would introduce when using the
  mean
\item
  Using Jan 2015 to Dec 2019 data, we fitted several different models
  and assessed their accuracy using a cross-validated process to reduce
  the likelihood of overfitting models to the data.

  \begin{itemize}
  \tightlist
  \item
    Started with 36 months of data, used 6-month horizon as this was the
    horizon we would be using for the forecasts, 3-month step (to keep
    run-times manageable)

    \begin{itemize}
    \tightlist
    \item
      Only interested in one forecast horizon: six-months
    \end{itemize}
  \end{itemize}
\end{itemize}

\hypertarget{all-prescribing}{%
\subsection{All prescribing}\label{all-prescribing}}

\hypertarget{excluded-practices}{%
\subsection{Excluded practices}\label{excluded-practices}}

Table with a row for each drug and a column for each reason a practice
could be excluded, with a total column at the end

\begin{enumerate}
\def\labelenumi{(\alph{enumi})}
\tightlist
\item
  missing any data in the COVID months, (b) more than 10\% of the
  pre-COVID data (i.e., 6 months) missing, (c) significant changes to
  prescribing in the six months pre-COVID (e.g., changes that suggested
  the practice had closed), or (d) MAPE \textgreater50.
\end{enumerate}

\begin{table}

\caption{(\#tab:excluded-practices-tab)Summary of GP practices removed for each drug.}
\centering
\begin{threeparttable}
\begin{tabular}[t]{lrrrrrr}
\toprule
drug & n & miss\_covid & miss\_10\_non\_covid & miss\_either & mape\_over\_50 & total\_removed\\
\midrule
ace & 531 & 6 & 1 & 7 & 0 & 7\\
adreno & 545 & 9 & 2 & 10 & 12 & 22\\
all\_drugs & 643 & 2 & 1 & 3 & 3 & 6\\
azithro & 488 & 26 & 29 & 41 & 85 & 126\\
contra & 494 & 9 & 5 & 10 & 4 & 14\\
cortico & 537 & 9 & 2 & 10 & 0 & 10\\
hcq & 480 & 13 & 15 & 22 & 14 & 36\\
nsaid & 537 & 9 & 2 & 10 & 1 & 11\\
oac & 511 & 9 & 3 & 11 & 0 & 11\\
paracet & 559 & 9 & 1 & 10 & 1 & 11\\
ssri & 543 & 9 & 1 & 10 & 0 & 10\\
vitd & 509 & 9 & 2 & 10 & 0 & 10\\
\bottomrule
\end{tabular}
\begin{tablenotes}
\item \textit{Note: } 
\item The total is not always equal to the row sum, as a single practice may be missing both data in covid months and more than 10\textbackslash{}\% of data in the non-covid months, but will only be excluded once.
\end{tablenotes}
\end{threeparttable}
\end{table}

\begin{verbatim}
## # A tibble: 1 x 1
##   `sum(n)`
##      <int>
## 1     6377
\end{verbatim}

\begin{verbatim}
## [1] 412
\end{verbatim}

\hypertarget{cross-validated-model-accuracy}{%
\subsection{Cross-validated model
accuracy}\label{cross-validated-model-accuracy}}

\begin{table}

\caption{(\#tab:cv-model-accuracy-by-drug-tab)Accuaracy of cross-validated models retained for forecasting.}
\centering
\begin{tabular}[t]{l|r|r|r}
\hline
drug & n\_practice\_retained & mape\_mean & mape\_sd\\
\hline
ace & 405 & 4.56 & 3.98\\
\hline
adreno & 390 & 18.66 & 7.37\\
\hline
all\_drugs & 406 & 6.86 & 3.61\\
\hline
azithro & 286 & 26.83 & 10.05\\
\hline
contra & 398 & 19.97 & 8.09\\
\hline
cortico & 402 & 8.75 & 5.64\\
\hline
hcq & 373 & 20.11 & 8.95\\
\hline
nsaid & 401 & 9.75 & 5.24\\
\hline
oac & 401 & 7.38 & 3.86\\
\hline
paracet & 401 & 6.40 & 3.95\\
\hline
ssri & 402 & 5.19 & 3.19\\
\hline
vitd & 402 & 7.71 & 4.50\\
\hline
\end{tabular}
\end{table}

\begin{figure}
\centering
\includegraphics{paper_files/figure-latex/individual-drug-accuracy-plot-1.pdf}
\caption{(\#fig:individual-drug-accuracy-plot)Accuracy of best model for
each regional unit by drug, as measured by mean absoloute percentage
error (MAPE). Models with MAPE \textgreater{} 50 have been removed.}
\end{figure}

\begin{figure}
\centering
\includegraphics{paper_files/figure-latex/best-cv-model-type-1.pdf}
\caption{(\#fig:best-cv-model-type)Performance by model type.}
\end{figure}

\newpage

\hypertarget{sm-references}{%
\subsection{SM References}\label{sm-references}}

\begingroup
\setlength{\parindent}{-0.5in}
\setlength{\leftskip}{0.5in}

\hypertarget{refs}{}
\leavevmode\hypertarget{ref-Box2015}{}%
Box, G. E. P., Jenkins, G. M., Reinsel, G. C. .., \& Ljung, G. M.
(2015). \emph{Time series analysis: forecasting and control} (5th ed.).
John Wiley \& Sons.

\leavevmode\hypertarget{ref-Chatfield1978}{}%
Chatfield, C. (1978). The Holt-Winters Forecasting Procedure.
\emph{Journal of the Royal Statistical Society: Series C (Applied
Statistics)}, \emph{27}(3), 264--279.

\leavevmode\hypertarget{ref-Clemen1989}{}%
Clemen, R. T. (1989). Combining forecasts: A review and annotated
bibliography. \emph{International Journal of Forecasting}, \emph{5}(4),
559--583. \url{https://doi.org/10.1016/0169-2070(89)90012-5}

\leavevmode\hypertarget{ref-Cleveland1990}{}%
Cleveland, R. B., Cleveland, W. S., McRae, J. E., \& Terpenning, I.
(1990). STL: A seasonal-trend decompostion procedure based on loess.
\emph{Journal of Official Statistics}, \emph{6}(1), 3--73.

\leavevmode\hypertarget{ref-Holt2004}{}%
Holt, C. C. (2004). Forecasting seasonals and trends by exponentially
weighted moving averages. \emph{International Journal of Forecasting},
\emph{20}(1), 5--10.
\url{https://doi.org/10.1016/j.ijforecast.2003.09.015}

\leavevmode\hypertarget{ref-Hyndman2008}{}%
Hyndman, R. J., \& Khandakar, Y. (2008). Automatic Time Series
Forecasting: The forecast Package for R. \emph{Journal of Statistical
Software}, \emph{27}(3), 1--22.
\url{http://www.jstatsoft.org/v27/i03/paper}

\leavevmode\hypertarget{ref-R-fable.prophet}{}%
O'Hara-Wild, M. (2020). \emph{Fable.prophet: Prophet modelling interface
for 'fable'}. \url{https://CRAN.R-project.org/package=fable.prophet}

\leavevmode\hypertarget{ref-R-fasster}{}%
O'Hara-Wild, M., \& Hyndman, R. (2021). \emph{Fasster: Fast additive
switching of seasonality, trend and exogenous regressors}.
\url{https://github.com/mitchelloharawild/fasster}

\leavevmode\hypertarget{ref-R-fable}{}%
O'Hara-Wild, M., Hyndman, R., \& Wang, E. (2021). \emph{Fable:
Forecasting models for tidy time series}.
\url{https://CRAN.R-project.org/package=fable}

\leavevmode\hypertarget{ref-Taylor2018}{}%
Taylor, S. J., \& Letham, B. (2018). Forecasting at Scale.
\emph{American Statistician}, \emph{72}(1), 37--45.
\url{https://doi.org/10.1080/00031305.2017.1380080}

\leavevmode\hypertarget{ref-Thomson2019}{}%
Thomson, M. E., Pollock, A. C., Önkal, D., \& Gönül, M. S. (2019).
Combining forecasts: Performance and coherence. \emph{International
Journal of Forecasting}, \emph{35}(2), 474--484.
\url{https://doi.org/10.1016/j.ijforecast.2018.10.006}

\leavevmode\hypertarget{ref-Winters1960}{}%
Winters, P. R. (1960). Forecasting Sales by Exponentially Weighted
Moving Averages. \emph{Management Science}, \emph{6}(3), 324--342.
\url{https://doi.org/10.1287/mnsc.6.3.324}

\endgroup

\newpage
\end{appendix}
